\documentclass{article}

\usepackage{amssymb}
\usepackage{amsfonts}
\usepackage{amsmath}
\usepackage[utf8]{inputenc}
\usepackage[spanish, mexico]{babel}

\usepackage[margin=0.5in]{geometry}

\title{Taller 1}
\author{Miguel A. Gomez B.}

\begin{document}
	\maketitle
\paragraph{1} Clasifique los siguientes experimentos en deterministas o aleatorios. Si es necesario, añada hipótesis o condiciones adicionales para justificar su respuesta.
\paragraph{a} Registrar el número de accidentes que ocurren en una determinada calle de una ciudad.
\paragraph{b} Observar la temperatura a la que hierve el agua a una altitud dada.
\paragraph{c} Registrar el consumo de electricidad de una casa-habitación en un día determinado.
\paragraph{d}Registrar la hora a la que desaparece el sol en el horizonte en un día dado, visto desde una posición geográfica determinada.
\paragraph{e} Observar el precio que tendrá el petróleo dentro de un año.
\paragraph{f} Registrar la altura máxima que alcanza un proyectil lanzado verticalmente.
\paragraph{g} Observar el número de años que vivirá un bebé que nace en este momento.
\paragraph{h} Observar el ángulo de reflexión de un haz de luz incidente en un espejo.
\paragraph{i} Registrar la precipitación pluvial anual en una zona geográfica determinada.
\paragraph{j} Observar el tiempo que tarda un objeto en caer al suelo cuando se le deja caer desde una altura dada.

\paragraph{2} Determine un espacio muestral para el experimento aleatorio consistente en:
\paragraph{a} Lanzar un dado hasta que se obtiene un "6".
\paragraph{b} Registrar la fecha de cumpleaños de n personas escogidas al azar.
\paragraph{c} Observar la forma en la que r personas que abordan un elevador en la planta baja de un edificio descienden en los pisos $1, 2, \dots, n$.
\paragraph{d} Registrar la duración de una llamada telefónica escogida al azar.
\paragraph{e} Observar el número de años que le restan de vida a una persona escogida al azar dentro del conjunto de asegurados de una compañía aseguradora.
\paragraph{3} Proponga un espacio muestral para el experimento aleatorio de lanzar tres monedas a un mismo tiempo, suponiendo que las monedas:
\paragraph{a} Son distinguibles, es decir, puede por ejemplo ser de colores distintos.
\paragraph{b}No son distinguibles, es decir, físicamente son idénticas.
\paragraph{4} Considere el experimento aleatorio de lanzar dos dados distinguibles. Escriba explícitamente los resultados asociados a los siguientes eventos y determine su cardinalidad.
\paragraph{a} $A = \text{'La suma de los dos resultados es 7'}$.
\paragraph{b} $B = \text{'Uno de los dos dados cae un número impar y el otro en un número par'}$.
\paragraph{c} $C = \text{'El resultado de un dado difiere del otro en, a lo sumo, una unidad'}$.
\paragraph{d} $D = \text{'El resultado de un dado difiere del otro en por lo menos cuatro unidades'}$.
\paragraph{e} $E = A \cap B$.
\paragraph{f} $F = B^{c}$.
\paragraph{e} $G = C \cup B$.
\end{document}